\documentclass[a4paper,12pt, czech]{article}
\usepackage[utf8]{inputenc} %Coding of input file - no problems with glyphs (ěš)
\usepackage[T1]{fontenc} %Coding of output file - no problems with glyphs (ěš)
\usepackage{babel} %Hyphenation and typesetting of loaded langs
\usepackage{datetime}

\usepackage{todonotes} %Adding todo notes
\usepackage{siunitx} %typography of units
\sisetup{separate-uncertainty=true,
	range-units = single}

\usepackage{caption}
\usepackage{subcaption}
\usepackage{graphicx} %Pictures

\usepackage{multirow} %Multi column and multirow tables
\usepackage{booktabs} %Better tables

\usepackage[pdftex,                %%% hyper-references for pdflatex
bookmarks=true,%                   %%% generate bookmarks ...
bookmarksnumbered=true,%           %%% ... with numbers
hypertexnames=false,%              %%% needed for correct links to figures !!!
breaklinks=true,%                  %%% break links if exceeding a single line
%linkbordercolor={0 0 1},
pdfauthor={Krystof Pilnacek},
pdftitle={Report z domaciho ukolu},
pdfkeywords={casove rady, akcie, obchodovany objem},
pdfsubject={Domaci ukol}]{hyperref}

\usepackage{xcolor} %Setting colors of references
\definecolor{dark-red}{rgb}{.4,.15,.15}
\definecolor{dark-blue}{rgb}{.15,.15,.4}
\definecolor{medium-blue}{rgb}{0,0,.5}
\hypersetup{
	colorlinks, linkcolor={dark-red},
	citecolor={dark-blue}, urlcolor={medium-blue} %url can be magenta 
}


\usepackage{parskip} %No indentation of first line on paragraph and line 
\usepackage{setspace} %Set spacing between lines
\setstretch{1.15}


\newcommand{\code}[1]{\texttt{#1}}


%===========Glossary================
\usepackage[acronym,toc,nopostdot, nonumberlist,nomain,shortcuts]{glossaries}%nogroupskip
%\usepackage{glossaries-extra}
\renewcommand{\glsgroupskip}{}% make nothing happen between groups 

%\newacronym{acr}{}{\vspace*{-7.65mm}}

\setacronymstyle{long-short}

\newacronym[description=Sezónní autoregresní integrovaný klouzavý průměr]{sarimax}{SARIMAX}{Seasonal Autoregressive Integrated Moving Average}
\newacronym[description=Vektorový autoregresní klouzavý průměr]{varmax}{VARMAX}{Vector Auto Regressive Moving Average}
\newacronym[description=Autoregrese]{ar}{AR}{Auto Regression}
\newacronym[description=Integrace]{i}{I}{Integration}
\newacronym[description=Klouzavý průměr]{ma}{MA}{Moving Average}

\makenoidxglossaries



%opening
\title{Report z domácího úkolu}
\author{Kryštof Pilnáček}

\begin{document}

\maketitle

\clearpage

\section{Zadání} \label{sec:zadani}

\begin{enumerate}
	\item Z Yahoo Finance stáhněte denní data akciového indexu S\&P 500 za období \formatdate{1}{1}{2010} -- \formatdate{31}{7}{2014}. \label{list:zadani_data}
	
	\item  Navrhněte a nakalibrujte několik různých modelů $E\left[v_{d+1}|F_d\right]$, kde $v_{d+1}$ je objem obchodů (volume) v den $d+1$ a $F_d$ je veškerá informace do dne $d$ (včetně). \label{list:zadani_modely}
	Odhadněte přesnost modelů (jako kritérium použijte $SSE$, resp. $R^2$) na testovacích (out of sample) datech a porovnejte tuto přesnost s referenčním modelem $\hat{E}\left[v_{d+1}\right] = v_d$.
	
\end{enumerate}

\section{Použité programy}

Kód pro analýzu časové řady byl vytvořen v programovacím jazyku \code{Python 3.6.1}.

Pro samotné zpracování a zobrazení dat byly navíc použity balíčky \code{numpy (1.13.0)}, \code{pandas (0.20.2)}, \code{matplotlib (2.0.2)} a \code{statsmodels (0.8.0)}.

\section{Vstupní data}

Vstupní byla ručně stažena z webu \href{https://finance.yahoo.com/quote/\%5EGSPC/history?period1=1262300400\&period2=1497045600\&interval=1d\&filter=history\&frequency=1d}{Yahoo Finance}.
Byly zkoušeny i další metody získávání dat z tohoto webu (balíčky Pythonu \code{yahoo-finance} a \code{pandas-datareader} či API Yahoo Finance), nicméně bez úspěchu.

Vstupní data jsou tedy staticky uložena v adresáři \code{./data/} v souboru nazvaném \code{\^{}GSPC.csv}.

\subsection{Zpracování dat}

Tabulková data byla načtena z \code{csv} souboru jako \code{pandas.DataFrame}.
Předzpracování proběhlo v následujících krocích:

\begin{enumerate}
	\item nastavení frekvence dat na pracovní dny,
	\item doplnění chybějících dat interpolací,
	\item rozdělení dat na testovací (out of sample) a kalibrační část (pro kalibraci použito 80\% dat),
	\item škálování dat do řádu $10^0$ (viz obr. \ref{fig:preprocess}), a
	\item variantně vytvoření první diference dat (viz obr. \ref{fig:preprocess}).
\end{enumerate}

\begin{figure}[htbp]
	\centering
	\includegraphics[width=0.85\linewidth]{../plots/preprocessing}
	\caption{Předzpracování použitých dat pomocí škálování (modrá řada) a následně první diference (oranžová řada)}
	\label{fig:preprocess}
\end{figure}

Dále byla provedena dekompozice sezónních vlivů pomocí funkce z \code{seasonal\_} \code{decompose} z balíčku \code{statsmodels} s přibližně čtvrtletní frekvencí (90 dní).
Výsledek je vidět na obrázku \ref{fig:seasonal_anl}.
Plyne z něj, že celkový trend objemu obchodů se ve sledovaném období lehce snižoval.
Dále je vidět, že nějaké sezónní vlivy se v časové řadě vyskytují nicméně jejich vliv je menší než vliv neznámých činitelů (v obrázku řada Residual).
Závěrem lze říci, že na sezónní vlivy bude muset při modelování být brán zřetel, protože se nejedná o stacionární řadu.

\begin{figure}[htbp]
	\centering
	\includegraphics[width=0.85\linewidth]{../plots/seasonal_anl}
	\caption{Dekompozice sezónních vlivů na objem obchodů}
	\label{fig:seasonal_anl}
\end{figure}


\section{Modely}

Pro modelování časových řad byly použity state space modely (modely stavového prostoru) \gls{sarimax} a \gls{varmax} z balíčku \code{statsmodels}.
První z modelů byl vybrán proto, že umožňuje popsat sezónní vlivy interně v modelu.
Druhý model byl vybrán proto, že dle bodu \ref{list:zadani_modely} v \nameref{sec:zadani} bylo nutno do modelu zahrnout veškerá dostupná data, tedy nejen předchozí data o objemech obchodů.

\subsection{Referenční model}

Referenční model byl sestaven podle bodu \ref{list:zadani_modely} v \nameref{sec:zadani}, tedy

\begin{equation}\label{eq:reference}
\hat{E}\left[v_{d+1}\right] = v_d.
\end{equation}

Konečná implementace vypadala tedy tak, že pro objem obchodů předpovídaných pro následující den byl dosazen objem obchodů ze dne současného, tedy $v_{d+1} = v_d$.
Pro škálovaná out of sample data vychází $R^2=\num{-.0297}$ a pro první diferenci škálovaných dat $R^2=\num{-1.7081}$.
V prvním případě je tedy předpověď tohoto modelu o něco málo horší než, kdybychom brali pouze průměr z těchto dat.
V případě druhém je výrazně horší.
V každém případě nám to poskytuje dobré porovnání pro méně naivní modely.

\subsection{\glsentryshort{sarimax}} \label{sec:sarimax}

Model \gls{sarimax} je modelem, který pro předpověď v následujícím dni používá pouze tu proměnnou, která je předpovídána, tedy 

\begin{equation}\label{eq:sarimax}
\hat{E}\left[v_{d+1}|v_d\right],
\end{equation}

což nesplňuje bod \ref{list:zadani_modely} v \nameref{sec:zadani}.
Nicméně je v něm možno postihnout sezónní vlivy i bez nutnosti úpravy vstupních dat.
Proto byl také v analýze ponechán.
Pro účely tohoto úkolu byla použita implementace z balíčku \code{statsmodels}.

Pro určení parametrů modelu pro \gls{ar} a \gls{ma} se běžně používají grafy autokorelačních a parciálních autokorelačních funkcí\footnote{viz např.: \href{https://www.analyticsvidhya.com/blog/2016/02/time-series-forecasting-codes-python/}{A comprehensive beginner’s guide to create a Time Series Forecast (with Codes in Python)}}.
Pro použitá data jsou grafy těchto funkcí vidět na obr. \ref{fig:auto_func}.

\begin{figure}[htb]
	\begin{subfigure}[t]{1\linewidth}
		\centering
		\includegraphics[width=1\linewidth]{../plots/acf}
		\caption{Autokorelační funkce}
		\label{fig:acf}
	\end{subfigure}
	\begin{subfigure}[t]{1\linewidth}
		\centering
		\includegraphics[width=1\linewidth]{../plots/pacf}
		\caption{Parciální autokorelační funkce}
		\label{fig:pacf}
	\end{subfigure}
	\caption{Autokorelační funkce použité pro určení \gls{ar} a \gls{ma} parametrů state space modelů.}\label{fig:auto_func}
\end{figure}

Z obr. \ref{fig:auto_func} vyplynulo, že by bylo vhodné použít parametry \gls{ar} a \gls{ma} $= 6$.
Hodnota parametru \gls{i} byla zvolena $= 1$, protože nás zajímá především první diference pro předpověď následujícího dne.
Dále byl do modelu zahrnut lineární trend s konstantou.
Tyto parametry byly vyzkoušeny nicméně optimalizace pro tyto parametry nedoběhla do konce.
Poté byly parametry použity následovně \gls{ar} $= 4$, \gls{i} $= 1$ a \gls{ma} $= 4$.
Výsledky jsou zobrazeny v obr. \ref{fig:sarimax}.
Výsledné koeficienty determinace pro škálovaná out of sample data vychází $R^2=\num{.2074}$ a pro první diferenci škálovaných dat $R^2=\num{-.0087}$.
V prvním i druhém případě tedy došlo ke zlepšení oproti referenčnímu modelu.

\begin{figure}[htb]
	\begin{subfigure}[t]{.5\linewidth}
		\centering
		\includegraphics[width=1.1\linewidth]{../plots/Model(SARIMAX)_scaled}
		\caption{Fit a predikce škálovaných dat}
		\label{fig:sarimax_scaled}
	\end{subfigure}
	\begin{subfigure}[t]{.5\linewidth}
		\centering
		\includegraphics[width=1.1\linewidth]{../plots/Model(SARIMAX)_first_diff}
		\caption{Fit a predikce škálovaných dat s první diferencí}
		\label{fig:sarimax_fd}
	\end{subfigure}
	\caption{Výsledky modelu \gls{sarimax}.}\label{fig:sarimax}
\end{figure}


\clearpage

\subsection{\glsentryshort{varmax}}


V modelu \gls{varmax} je pro předpověď bráno více proměnných. Model je také dokáže všechny předpovědět a tudíž dokáže splnit podmínku z bodu \ref{list:zadani_modely} v \nameref{sec:zadani}.

\begin{equation}\label{eq:varimax}
\hat{E}\left[v_{d+1}|F_d\right]
\end{equation}

Parametry použité pro výpočty v tomto úkolu byly následující \gls{ar} $= 4$, \gls{ma} $= 1$ s konstantním trendem.
Pro předpověď objemu obchodů (volume) byly použity hodnoty denní otevírací (open) a uzavírací ceny (close).
V principu by šlo použít i další proměnné, nicméně přidání hodnot denní nejvyšší a nejnižší ceny (high, low) nepřineslo zlepšení předpovědních schopností a tudíž nebyly pro rychlost výpočtů použity.
Pro účely tohoto úkolu byla použita implementace z balíčku \code{statsmodels}.

Výsledky jsou zobrazeny na obr. \ref{fig:varmax}.
Výsledné koeficienty determinace pro škálovaná out of sample data vychází $R^2=\num{.0985}$ a pro první diferenci škálovaných dat $R^2=\num{.1953}$.
Horší předpověď škálovaných data a podstatně lepší předpověď první diference škálovaných dat, ve srovnání s modelem \ref{sec:sarimax}, lze vysvětlit absencí proměnných popisujících sezónní vlivy.

\begin{figure}[htb]
	\begin{subfigure}[t]{.5\linewidth}
		\centering
		\includegraphics[width=1.1\linewidth]{../plots/Model(VARMAX)_scaled}
		\caption{Fit a predikce škálovaných dat}
		\label{fig:varmax_scaled}
	\end{subfigure}
	\begin{subfigure}[t]{.5\linewidth}
		\centering
		\includegraphics[width=1.1\linewidth]{../plots/Model(VARMAX)_first_diff}
		\caption{Fit a predikce škálovaných dat s první diferencí}
		\label{fig:varmax_fd}
	\end{subfigure}
	\caption{Výsledky modelu \gls{varmax}.}\label{fig:varmax}
\end{figure}

\clearpage

\section{Závěr}

Závěrem lze konstatovat, že oba body zadání byly splněny.
Povedlo se stáhnout data z Yahoo Finance pro dané časové období a vytvořit pro ně předpovědní model zahrnující více proměnných.
Celkové porovnání hodnot koeficientů determinace je uvedeno v tabulce \ref{tbl:r2}.


\begin{table} [htbp]
	\caption{Porovnání $R^2$ pro všechny uvažované modely}
	\label{tbl:r2}
	\begin{center}
		\begin{tabular}{c c S[round-precision = 4,]}
			\toprule
			Model    & Předzpracování 	& 	$R$\textsuperscript{2} 	 \\
			\midrule
			\multirow{2}{*}{Referenční}    & škálování 	&  -.0297 \\
			& + první dif.	& -1.7081 \\
			\multirow{2}{*}{\gls{sarimax}} & škálování   &   .2074 \\
			& + první dif.   &  -.0087 \\
			\multirow{2}{*}{\gls{varmax}}  & škálování   &   .0985 \\
			& + první dif.   &   .1953 \\
			\bottomrule
		\end{tabular}
	\end{center}
\end{table}

Pro zlepšení předpovědních schopností modelu \gls{varmax} by bylo možné použít ještě dalších informací, např. kalendáře očekávaných událostí pro daný den.
Z něj by mohly být vytvořeny tzv. dummy proměnné pro daný typ události a ty použity pro předpověď.
K dalšímu zlepšení by mohlo dojít optimalizací \gls{ar} a \gls{ma} parametrů modelu, případně zlepšením předúpravy vstupních dat odstraněním sezónnosti.

\clearpage
\singlespacing
\printnoidxglossary[type=acronym,title=Seznam zkratek,sort=word]


\end{document}
