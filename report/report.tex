\documentclass[a4paper,12pt, czech]{article}
\usepackage[utf8]{inputenc} %Coding of input file - no problems with glyphs (ěš)
\usepackage[T1]{fontenc} %Coding of output file - no problems with glyphs (ěš)
\usepackage{babel} %Hyphenation and typesetting of loaded langs
\usepackage{datetime}

\usepackage{caption}
\usepackage{graphicx} %Pictures

\usepackage[pdftex,                %%% hyper-references for pdflatex
bookmarks=true,%                   %%% generate bookmarks ...
bookmarksnumbered=true,%           %%% ... with numbers
hypertexnames=false,%              %%% needed for correct links to figures !!!
breaklinks=true,%                  %%% break links if exceeding a single line
%linkbordercolor={0 0 1},
pdfauthor={Krystof Pilnacek},
pdftitle={Report z domaciho ukolu},
pdfkeywords={casove rady, akcie, obchodovany objem},
pdfsubject={Domaci ukol}]{hyperref}

\usepackage{xcolor} %Setting colors of references
\definecolor{dark-red}{rgb}{.4,.15,.15}
\definecolor{dark-blue}{rgb}{.15,.15,.4}
\definecolor{medium-blue}{rgb}{0,0,.5}
\hypersetup{
	colorlinks, linkcolor={dark-red},
	citecolor={dark-blue}, urlcolor={medium-blue} %url can be magenta 
}


\usepackage{parskip} %No indentation of first line on paragraph and line 
\usepackage{setspace} %Set spacing between lines
\setstretch{1.15}


\newcommand{\code}[1]{\texttt{#1}}

%opening
\title{Report z domácího úkolu}
\author{Kryštof Pilnáček}

\begin{document}

\maketitle

\clearpage

\section{Zadání}

\begin{enumerate}
	\item Z Yahoo Finance stáhněte denní data akciového indexu S\&P 500 za období \formatdate{1}{1}{2010} -- \formatdate{31}{7}{2014}.
	
	\item  Navrhněte a nakalibrujte několik různých modelů $E\left[v_{d+1}|F_d\right]$, kde $v_{d+1}$ je objem obchodů (volume) v den $d+1$ a $F_d$ je veškerá informace do dne $d$ (včetně).
	Odhadněte přesnost modelů (jako kritérium použijte $SSE$, resp. $R^2$) na testovacích (out of sample) datech a porovnejte tuto přesnost s referenčním modelem $\hat{E}\left[v_{d+1}\right] = v_d$.
	
\end{enumerate}

\section{Použité programy}

Kód pro analýzu časové řady byl vytvořen v programovacím jazyku \code{Python 3.6.1}.

Pro samotné zpracování a zobrazení dat byly navíc použity balíčky \code{numpy (1.13.0)}, \code{pandas (0.20.2)}, \code{matplotlib (2.0.2)} a \code{statsmodels (0.8.0)}.

\section{Vstupní data}

Vstupní byla ručně stažena z webu \href{https://finance.yahoo.com/quote/\%5EGSPC/history?period1=1262300400\&period2=1497045600\&interval=1d\&filter=history\&frequency=1d}{Yahoo Finance}.
Byly zkoušeny i další metody získávání dat z tohoto webu (balíčky Pythonu \code{yahoo-finance} a \code{pandas-datareader} či API Yahoo Finance), nicméně bez úspěchu.

Vstupní data jsou tedy staticky uložena v adresáři \code{./data/} v souboru nazvaném \code{\^{}GSPC.csv}.

\subsection{Zpracování dat}

Tabulková data byla načtena z \code{csv} souboru jako \code{pandas.DataFrame}.
Předzpracování proběhlo v následujících krocích:

\begin{enumerate}
	\item nastavení frekvence dat na pracovní dny,
	\item doplnění chybějících dat interpolací,
	\item rozdělení dat na testovací (out of sample) a kalibrační část (pro kalibraci použito 80\% dat),
	\item škálování dat do řádu $10^0$ (viz obr. \ref{fig:preprocess}), a
	\item variantně vytvoření první diference dat (viz obr. \ref{fig:preprocess}).
\end{enumerate}

\begin{figure}[htbp]
	\centering
	\includegraphics[width=0.85\linewidth]{../plots/preprocessing}
	\caption{Předzpracování použitých dat pomocí škálování (modrá řada) a následně první diference (oranžová řada)}
	\label{fig:preprocess}
\end{figure}


\section{Modely}



\subsection{Referenční model}

\subsection{SARIMAX}

\subsection{VARMAX}

\clearpage

\section{Závěr}

\end{document}
